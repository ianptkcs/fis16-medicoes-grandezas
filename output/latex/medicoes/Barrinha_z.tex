\documentclass{article}
\usepackage{amsmath}
\usepackage{booktabs}
\usepackage[portuguese]{babel}
\usepackage[a4paper, margin=1.5cm]{geometry}
\begin{document}

\begin{table}[h!]
\centering
\begin{tabular}{c c c c c c }
\toprule
Medição nº & \shortstack{Nelson\\Micrômetro\\\textit{(± 0,025 mm)}} & \shortstack{Patrick\\Micrômetro\\\textit{(± 0,025 mm)}} & \shortstack{Gabriel\\Micrômetro\\\textit{(± 0,025 mm)}} & \shortstack{Ian\\Micrômetro\\\textit{(± 0,025 mm)}} & \shortstack{Henrique\\Micrômetro\\\textit{(± 0,025 mm)}}\\
\midrule
1 & 2,500 & 2,350 & 2,600 & 2,400 & 2,450\\
2 & 2,450 & 2,300 & 2,550 & 2,350 & 2,400\\
3 & 2,450 & 2,300 & 2,550 & 2,350 & 2,400\\
4 & 2,500 & 2,350 & 2,600 & 2,400 & 2,450\\
5 & 2,550 & 2,400 & 2,650 & 2,450 & 2,500\\
6 & 2,400 & 2,250 & 2,500 & 2,300 & 2,350\\
7 & 2,500 & 2,350 & 2,600 & 2,400 & 2,450\\
8 & 2,400 & 2,250 & 2,500 & 2,300 & 2,350\\
9 & 2,400 & 2,250 & 2,500 & 2,300 & 2,350\\
10 & 2,450 & 2,300 & 2,550 & 2,350 & 2,400\\
\midrule
&\shortstack{Nelson\\(Micrômetro)\\\textit{(mm)}} & \shortstack{Patrick\\(Micrômetro)\\\textit{(mm)}} & \shortstack{Gabriel\\(Micrômetro)\\\textit{(mm)}} & \shortstack{Ian\\(Micrômetro)\\\textit{(mm)}} & \shortstack{Henrique\\(Micrômetro)\\\textit{(mm)}}\\
\midrule
Média & 2,460 & 2,310 & 2,560 & 2,360 & 2,410\\[3pt]
\shortstack{Desvio\\Padrão da\\Média} & 0,016 & 0,016 & 0,016 & 0,016 & 0,016\\[3pt]
Incerteza & 0,030 & 0,030 & 0,030 & 0,030 & 0,030\\
\bottomrule
\end{tabular}
\end{table}
\end{document}
