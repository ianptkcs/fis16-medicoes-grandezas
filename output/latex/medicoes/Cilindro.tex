\documentclass{article}
\usepackage{amsmath}
\usepackage{booktabs}
\usepackage[portuguese]{babel}
\usepackage[a4paper, margin=0.1cm, top=0.5cm, bottom=0.5cm]{geometry}
\begin{document}

\begin{table}[h!]
\centering
\begin{tabular}{c c c c c c }
\toprule
Medição nº & \shortstack{Nelson\\Paquímetro\\\textit{(± 0,05 mm)}} & \shortstack{Patrick\\Régua\\\textit{(± 5,0 mm)}} & \shortstack{Gabriel\\Trena\\\textit{(± 5,0 mm)}} & \shortstack{Ian\\Paquímetro\\\textit{(± 0,05 mm)}} & \shortstack{Henrique\\Régua\\\textit{(± 5,0 mm)}}\\
\midrule
1 & 126,10 & 127,0 & 126,0 & 126,20 & 126,0\\
2 & 126,30 & 126,0 & 125,0 & 126,10 & 127,0\\
3 & 126,10 & 127,0 & 127,0 & 126,30 & 125,0\\
4 & 126,20 & 125,0 & 126,0 & 126,10 & 126,0\\
5 & 126,20 & 126,0 & 127,0 & 126,20 & 127,0\\
6 & 126,10 & 127,0 & 126,0 & 126,20 & 126,0\\
7 & 126,30 & 126,0 & 125,0 & 126,10 & 125,0\\
8 & 126,10 & 127,0 & 127,0 & 126,30 & 127,0\\
9 & 126,20 & 125,0 & 126,0 & 126,10 & 126,0\\
10 & 126,20 & 126,0 & 127,0 & 126,20 & 125,0\\
\midrule
&\shortstack{Nelson\\(Paquímetro)\\\textit{(mm)}} & \shortstack{Patrick\\(Régua)\\\textit{(mm)}} & \shortstack{Gabriel\\(Trena)\\\textit{(mm)}} & \shortstack{Ian\\(Paquímetro)\\\textit{(mm)}} & \shortstack{Henrique\\(Régua)\\\textit{(mm)}}\\
\midrule
Média & 126,18 & 126,2 & 126,2 & 126,18 & 126,0\\[3pt]
\shortstack{Desvio\\Padrão da\\Média} & 0,02 & 0,2 & 0,2 & 0,02 & 0,3\\[3pt]
Incerteza & 0,06 & 5,0 & 5,0 & 0,06 & 5,0\\
\bottomrule
\end{tabular}
\end{table}
\end{document}
