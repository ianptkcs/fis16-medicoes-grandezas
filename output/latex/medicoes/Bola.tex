\documentclass{article}
\usepackage{amsmath}
\usepackage{booktabs}
\usepackage[portuguese]{babel}
\usepackage[a4paper, margin=1.5cm]{geometry}
\begin{document}

\begin{table}[h!]
\centering
\begin{tabular}{c c c c c c }
\toprule
Medição nº & \shortstack{Nelson\\Paquímetro\\\textit{(± 0,05 mm)}} & \shortstack{Patrick\\Micrômetro\\\textit{(± 0,25 mm)}} & \shortstack{Gabriel\\Paquímetro\\\textit{(± 0,05 mm)}} & \shortstack{Ian\\Paquímetro\\\textit{(± 0,05 mm)}} & \shortstack{Henrique\\Micrômetro\\\textit{(± 0,25 mm)}}\\
\midrule
1 & 17,90 & 17,84 & 18,20 & 18,00 & 17,86\\
2 & 18,00 & 17,80 & 18,00 & 17,90 & 17,84\\
3 & 17,80 & 17,82 & 17,90 & 18,10 & 17,80\\
4 & 18,10 & 17,84 & 18,10 & 18,20 & 17,82\\
5 & 18,20 & 17,86 & 18,20 & 18,00 & 17,84\\
6 & 18,00 & 17,84 & 18,00 & 17,90 & 17,86\\
7 & 18,00 & 17,80 & 17,90 & 18,10 & 17,84\\
8 & 17,80 & 17,82 & 18,10 & 18,20 & 17,80\\
9 & 18,10 & 17,84 & 18,20 & 18,00 & 17,82\\
10 & 18,20 & 17,86 & 18,00 & 17,90 & 17,84\\
\midrule
&\shortstack{Nelson\\(Paquímetro)\\\textit{(mm)}} & \shortstack{Patrick\\(Micrômetro)\\\textit{(mm)}} & \shortstack{Gabriel\\(Paquímetro)\\\textit{(mm)}} & \shortstack{Ian\\(Paquímetro)\\\textit{(mm)}} & \shortstack{Henrique\\(Micrômetro)\\\textit{(mm)}}\\
\midrule
Média & 18,01 & 17,83 & 18,06 & 18,03 & 17,83\\[3pt]
\shortstack{Desvio\\Padrão da\\Média} & 0,05 & 0,01 & 0,04 & 0,04 & 0,01\\[3pt]
Incerteza & 0,07 & 0,25 & 0,06 & 0,06 & 0,25\\
\bottomrule
\end{tabular}
\end{table}
\end{document}
